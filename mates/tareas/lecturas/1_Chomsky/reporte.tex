\documentclass[11pt,a4paper]{article}
    \usepackage[utf8]{inputenc}
    \usepackage[margin=1in]{geometry}
    \usepackage{amsmath}
    \usepackage{amssymb}
    
        \title{Reporte Sobre Lectura 1}
        \author{Gerardo Galván Olvera A01371872}
        \date{22 de agosto de 2017}

        \begin{document}
            \maketitle
            Lectura: \textit{N. Chomsky. Three models for the description of languages. IRE Transactions on information theory, vol. 2(3), pp. 113-124 (1956)}

            \begin{itemize}
                \item Text's main idea\\
                    \textit{Three models for the description of language} is an interesting and (in my view) original attempt at formalizing the definiton of the word \textbf{language}. Formalizing in the sense that the level of abstraction he uses is, by any means, mathematical. He specifically sets the test subject language of his research to be the English Language. As such, he faces the challenge of standarizing and describing an organic system that is fuzzy and rampant in a way that is not easy to describe.
                   
                \item Main ideas' summary\\
                In the trying to find a theory of the structure of language, three different approaches are presented:
                    \begin{itemize}
                        \item Finite state Markov processes: A system that produces strings using dependencies as rules.
                        \item Phrase Structure: Tree-like composition of sentences.
                        \item Transformational Grammar: Base language kernel from wich all sentences come through transformations.
                    \end{itemize}
                \item Own comment on interesting ideas\\
                    The most interesting aspect of this work was, for me, also the most intriguing. That was the way in which these possible theories of language (or the english language) came to be. That is to say: my main question would be how were the basis of these theories constructed in anyone's mind. They are presented in a way that sounds as if they emanated out of language itself, or its usage when in reality they do not in my eyes. I feel like an introduction to and exploration of the \textit{actual} rules and structure of language (even if it is not mathematically solid) would aid the understanding of this work.
            \end{itemize}
    
        \end{document}